\documentclass[titlepage, a4paper, 12pt]{article}
\usepackage{listings}
\usepackage{color}
\usepackage[pdftex]{graphicx}
\usepackage{graphicx}
\usepackage[colorlinks=false]{hyperref}
\lstset{ %
language=sh,
basicstyle=\footnotesize,
numbers=none,
backgroundcolor=\color{white},
showspaces=false,
showstringspaces=false,
showtabs=false,
frame=single,
tabsize=2,
captionpos=b,
breaklines=true,
breakatwhitespace=false,
title=\lstname
}
\title{Apropos Replacement \newline
(Initial Outline)}
\author{Abhinav Upadhyay $<$er.abhinav.upadhyay@gmail.com$>$ \and 
Joerg Sonnenberger $<$joerg@NetBSD.org$>$}
\begin{document}
\maketitle
\section{Abstract}
Unix like operating systems have withstood the test of time for 40 odd years for
several technical and philosophical reasons. One of those reasons is the
quality documentation they ship in the form of man pages. They are a single point
source of consultation for most of our routine work. But there has been a lack of
good search tool to accompany this documentation which would increase their
usefulness manifolds.

Traditionally \textit{apropos} [1] has been there to act as a search
interface but to a large extent it hasn't really lived up to it's expectation.
\textit{apropos} was
developed in the early days of Unix when computing resources were scarce and
which is the primary reason that it's design was kept so
minimal and simple.

In this paper the limitations of the traditional implementation of
\textit{apropos} are discussed along with brief details of how those limitations
have been solved in the implementation developed as part of this project and also
what other modern implementations are available out there.
\section{Introduction}
The classical version of \textit{apropos} has been implemented by simply
indexing the keywords in the NAME section of the man pages in a plain text file
(whatis.db) [2]
and performing searches on it. The reason for this simple design was most
probably lack of computing resources in the early days. The plain text file
consisting of keywords would hardly take few hundreds of kilo bytes of disk space
and performing search on a plain text file is also not very hard.

This simplified design of \textit{apropos(1)} resulted in limited utility and
usability.The searches are limited to the keywords in the NAME section of the
 keywords. If
only a user knows the exact keywords he gets the right results otherwise most of
the times the searches are full of bogus results or a dead end. In the modern
computing world where hard problems like searching the World Wide Web have been
solved [3] to sufficient degree where most of the data is unstructured, then it
makes perfect sense that we leverage the advancement in technology to solve the
problem of an effective man page search tool, considering the fact that the man
pages consist of structured data.

In present times the machines are powerful and  capable of running the search
algorithms efficiently. Disk
space is also cheap enough that indexing of additional metadata from man pages
can be easily afforded. These are necessary prerequisites for building an
effective search tool.

\section{Limitations of Conventional \textit{apropos(1)}}
In this section results of some sample query searches from
the classical version of \textit{apropos} are shown and their shortcomings are
analyzed.

\subsection{Lack of support for free form queries}
The conventional \textit{apropos} is limited to the keywords used in the NAME
section
of the man pages. If the user specifies keywords relevant to the page which he
is looking but probably those keywords are not used in it's NAME section,
then \textit{apropos(1)} would not return
anything relevant. So in essence the user cannot use free form queries like
those supported by modern search applications.
\lstset{language = sh, caption={No results for free form queries}}
\begin{lstlisting}
$ apropos(1) ``add new user''
add new user: nothing appropriate

$ apropos(1) ``get process status''
get process status: nothing appropriate

$ apropos(1) ``termcap database''
termcap database: nothing appropriate

$ apropos(1) ``signal number to string''
signal number to string: nothing appropriate
\end{lstlisting}

\subsection{Lack of basic language support}
Another major limitation of the classical version of \textit{apropos} is that
it has no
natural language support. For example in the following listing it can be seen
that while \textit{apropos} returns the correct result for the query \textit{``make directories''} but it fails when the keyword \textit{``directory''} is
used in place of \textit{``directories''}, although the two words are based on
same root word.
\lstset{language = sh, caption={No support for stems or word with same roots},}
\begin{lstlisting}
$ apropos ``make directories''
mkdir (1) - make directories
$ apropos ``make directory''
make directory: nothing appropriate

$ apropos "upgrading software package"
pkg_add (1) - a utility for installing and upgrading software package
distributions
$ apropos "upgrading software packages"
upgrading software packages: nothing appropriate
\end{lstlisting}

Similarly, it is very common for users to misspell keywords in a query and \textit{apropos} has no support for it either.
\lstset{language = sh, caption={No spelling correction},}
\begin{lstlisting}
$ apropos ``copy strings''
stpcpy, stpncpy, strcpy, strncpy (3) - copy strings
$ apropos ``coppy strings''
coppy strings: nothing appropriate
\end{lstlisting}

\subsection{Unintelligible Output}
Because of the way \textit{apropos} works, sometimes it's output can be
unintelligible and it can be very hard for the user to identify the results which
are relevant to him.
\lstset{language = sh, caption={Unintelligible output of \textit{apropos(1)}},}
\begin{lstlisting}
$ \textit{apropos(1)} power
PCI, pci_activate, pci_bus_devorder, pci_chipset_tag_create,
pci_chipset_tag_destroy, pci_conf_read, pci_conf_write,
pci_conf_print, pci_conf_capture, pci_conf_restore, pci_find_device,
pci_get_capability, pci_mapreg_type, pci_mapreg_map, pci_mapreg_info,
pci_intr_map, pci_intr_string, pci_intr_evcnt, pci_intr_establish,
pci_intr_disestablish, pci_get_powerstate, pci_set_powerstate,
pci_vpd_read, pci_vpd_write, pci_make_tag, pci_decompose_tag,
pci_findvendor, pci_devinfo, PCI_VENDOR, PCI_PRODUCT, PCI_REVISION (9)
- Peripheral Component Interconnect
PMF, pmf_device_register, pmf_device_register1, pmf_device_deregister,
pmf_device_suspend, pmf_device_resume, pmf_device_recursive_suspend,
pmf_device_recursive_resume, pmf_device_resume_subtree,
pmf_class_network_register, pmf_class_input_register,
pmf_class_display_register, pmf_system_suspend, pmf_system_resume,
pmf_system_shutdown, pmf_event_register, pmf_event_deregister,
pmf_event_inject, pmf_set_platform, pmf_get_platform (9) - power
management and inter-driver messaging framework
acpi (4) - Advanced Configuration and Power Interface
acpipmtr (4) - ACPI Power Meter
amdpm (4) - AMD768 Power Management Controller and AMD8111 System
Management Controller
...
.
.
.
\end{lstlisting}
\subsection{Other Problems}
Apart from the search related problems there are a few issues related to
the way man pages are handled in NetBSD. The different aliases of the man pages
are stored on the filesystem in the from of hard (or soft) links and these
have to be mentioned in the makefiles explicitly using the MLINKS mechanism. This
approach works fine but it is a mess from maintenance point of view.
It should be possible to fix this by utilizing the index already built and
maintained by \textit{apropos} and related tools, but yet again, the
simplistic implementation of \textit{apropos} does not leave any room for
improvement.

\section{Proposed Solution}
A very simple and straightforward solution is proposed to solve these problems.
The idea is to parse and index
the complete content (or at least most of the content) in the form of an
inverted index [4] and use it to build a full text search
interface. Having an index based on the complete content of the man pages solves
many of the search
related problems associated with the conventional \textit{apropos(1)} as noted
before and discussed as follows:
\begin{description}
\item[Free Form Queries] \hfill \\
The users can express their queries in more natural language form as the searches
are no longer limited to the keywords defined in the NAME section of the man
pages, the keywords used throughout the body of the page also play a role.
For example queries like \textit{``installing new software package''}
will now get results.
\end{description}
\begin{description}
\item[Basic Natural Language Processing Support] \hfill \\
When parsing the man pages and building the index, it is also possible to
pre-process the tokens extracted from the man pages to support some of the very
basic natural language processing functionalities. In this implementation the
Porter stemming algorithm [5] has been used to
reduce the individual tokens extracted from the man pages to their root words.
This enables support for more flexible searches. For example both
\textit{``Installing new packages''} and \textit{``install new package''} will
return same results.

Similarly along with the inverted index, a dictionary of the keywords frequently
occurring in the corpus of man pages can also be built and used to support
spelling suggestion.

In this implementation, both word stemming and spelling correction in our implementation are used to make the search experience more smooth.
\end{description}
\begin{description}
\item[Bookkeeping of man page metadata] \hfill \\
In this implementation additional metadata related to the man pages, for example
an md5 hash, the device id, inode number and modification time
of the man page file etc. are indexed and stored so
as to support fast and hassle free updating of the index as new man pages are
installed or the old ones are modified/updated.

Similarly a separate index of all the man page aliases is stored and maintained.
This provides an option to get rid of all the hard or symbolic links of the man
pages scattered throughout the filesystem, and also for clean up of the MLINKS
mess in the makefiles.
\end{description}
\subsection{Tools Used}
The two main tasks that are needed to be performed are parsing of the man pages
and building an index of the data. \textit{libmandoc} [6]
and \textit{sqlite} [7] have been used for the two tasks respectively in this
implementation.
\begin{description}
\item[\textit{libmandoc}] \hfill \\
\textit{libmandoc} is a library interface to a validating compiler. It provides
interface to parse and build an AST (Abstract Syntax Tree) of the man page, 
and interface for traversing that tree in order to extract the data from nodes
which are of our interest.
\end{description}
\begin{description}
\item[\textit{sqlite}] \hfill \\
\textit{sqlite} is an embedded relational database management system providing a
relatively small and easy to use C library interface. One of the main reason
for choosing it over the myriad of other possible options is that it provides
in built support for full text search through it's FTS virtual table module [8].
The FTS module can be accessed using pretty much standard SQL syntax, and it is
still flexible enough to accept user supplied ranking function to suit the needs
of the application. Besides that another advantage of \textit{sqlite} is the
RDBMS support which makes it very easy to store additional metadata in the form
of normal database tables without any hassles.
\end{description}
\section{Implementation Details}
Due to space constraints it is not possibile to go into enough implementation details but in this section some important components of this project are
discussed in brief.
\begin{description}
\item[makemandb] \hfill \\
\textit{makemandb} [9] is the key component of this implementation. It is a
command line tool which
traverses the filesystem, reads the raw man page source files, feeds them to the
\textit{libmandoc} parser and then stores the extracted data in the database
using \textit{sqlite}.
\end{description}
\begin{description}
\item[apropos] \hfill \\
This is the version of \textit{apropos} written from scratch utilizing the
full text search functionality of \textit{sqlite}. Unlike the classical version
of \textit{apropos}, it only displays the top 10 results relevant to the user
query and most of the times this is sufficient as will be seen later. Also, it
shows a brief snippet for each of the search results, making it more easy to
identify the relevant documents.
\end{description}
\begin{description}
\item[apropos-utils] \hfill \\
\textit{apropos-utils} [10] is a small library interface provided with this
implementation. It provides functions for querying the FTS index and for
processing the results in a user supplied callback function. It's main purpose
is to develop different interfaces on top of this implementation for different
use cases. For example a small CGI application was built using it for doing
the searches from a web browser, similarly an IRC bot was also developed utilizing this interface.
\end{description}
\section{Results}
This section shows results of some of the sample queries on the version of
developed as part of this project
\textit{apropos} to demonstrate how it solves many of the problems cited earlier
with the classical version of \textit{apropos}.
\footnote{Although this implementation of \textit{apropos(1)} returns 10 results
in normal case for a query but in the following listings the output has been
snipped to save space
}
\lstset{language = sh, caption={Add new user},}
\begin{lstlisting}
$ apropos ``add new user''
ssh-add(1)      adds private key identities to the authentication agent
...on the command line. If any file requires a passphrase, ssh-add
asks for the passphrase from the user. The passphrase is read from the
user's tty. ssh-add retries the last passphrase if multiple identity
files are given...

chpass(1)       add or change user database information
add or change user database information

useradd(8)      add a user to the system
The useradd utility adds a user to the system, creating and populating
a home directory if necessary. Any skeleton files will be provided for
the new user if they exist in the skel-dir directory (see the k
option). Default...
.
.
.
\end{lstlisting}
\lstset{language = sh, caption={make directory},}
\begin{lstlisting}
$ apropos ``make directory''
make(1) maintain program dependencies
...CURDIR A path to the directory where make was executed. Refer to
the description of PWD for more details. MAKE The name that make was
executed with argv[0] . For compatibility make also sets .MAKE with
the same value. The...

mkdir(1)        make directories
make directories

ln(1)   make links
...a directory in which to place the link; otherwise it is placed in
the current directory. If only the directory is specified, the link
will be made to the last component of source_file . Given more than
two arguments, ln makes...

mkfifo(1)       make fifos
make fifos...of a=rw mkfifo requires write permission in the parent
directory. mkfifo exits 0 if successful, and >0 if an...

mkdir(2)        make a directory file
...will contain the directory has been exhausted. EDQUOT The user's
quota of inodes on the file system on which the directory is being
created has been exhausted. EIO An I/O error occurred while making the
directory entry or...
\end{lstlisting}
\lstset{language = sh, caption={signal number to string},}
\begin{lstlisting}
$ apropos ``signal number to string''
psignal(3)      system signal messages
...the signal number is not recognized sigaction(2) , the string
Unknown signal is produced. The psiginfo function produces the same
output as the psignal function, only it uses the signal number
information from the si argument. The message strings can...

intro(2)        introduction to system calls and error numbers
...undefined signal to a signal(3) or kill(2) function). 23 ENFILE Too
many open files in system . Maximum number...shell. Pathname A path
name is a NUL -terminated character string starting with an optional
slash \&/ , followed by zero or...

groff_mdoc(7)   reference for groff's mdoc implementation
...Qq .Qq string ) , string ) , .Qq string Ns ), string ), .Sq .Sq
string string .Em or...UNTITLED is used. The section number may be a
number in the range 1...2, 3 and 9 error \&.\e"     and signal
handling only. \&.\e" .Sh ERRORS \&.\e...
\end{lstlisting}

\begin{figure}[htp]
\includegraphics[scale=0.30]{/home/abhinav/development/AsiaBSDCon/spell.png}
\caption{The spell corrector and the web interface in action}
\label{}
\end{figure}
\section{Related Work}
There are at least two projects which are related to this in some way.
\begin{description}
\item[man-db] \hfill \\
\textit{man-db} [11] is a complete implementation of the man page documentation
system and it is used on a number of GNU/Linux distributions.\textit{man-db}
takes an interesting approach for indexing the man page data. Unlike the
classical \textit{apropos}, it uses a Berkley DB database but still it's index
is limited to the NAME section only. It adds an option `K' to man(1) to allow a
crude full text search but it is not very efficient nor effective.
\end{description}
\begin{description}
\item[mandocdb] \hfill \\
\textit{mandocdb} [12] is more in line with the goals of our project but it takes
a novel approach. It indexes the keywords extracted from the man pages in a
key-value store using btree(3) [13]. It also comes with it's own implementation
of \textit{apropos} which performs search using this key-value store. The
main key point of this implementation is that it exploits the semantic structure
of the man pages. 
\end{description}
\section{Future Work}
\begin{description}
\item[Work on Ranking Algorithm] \hfill \\
A ranking algorithm based on probabilistic model [14] of information
retrieval has been implemented to filter out the most relevant results and show
them at the top. It is essentially based on the Okapi BM25F algorithm [15] and
uses certain parameters whose values are usually dependent on the corpus and the
search application. For example certain weight parameters are assigned to different sections of man pages. At the moment these values have been determined
manually but one goal is to use some machine learning techniques in the coming
days to automate this.
\end{description}
\begin{description}
\item[Fix Some Loose Ends In Parsing of man pages] \hfill \\
Although the parsing routine is working fine for most purposes on the man pages
supplied with NetBSD and also on most of the man pages found in Pkgsrc. But there
are perhaps a few corner cases still to be fixed. Besides that another issue that
is to be addressed is the parsing of the escape sequences which are used
extensively in the \textit{man(7)} [16] based man pages. These issues are to be
addressed in the coming days.
\end{description}
\section{Acknowledgement}
This project has been developed as part of \textbf{Google Summer of Code 2011}
[17], so thanks to \textbf{Google} for sponsoring it. Special thanks to
\textbf{Kristaps Dzonsons} who is the developer of the \textit{mdocml} [18]
project, he also helped by pointing out several issues in the parsing related
code. We would like to thank \textbf{David Young} who was involved with
this project closely and offered useful help and guidance throughout. A special
thanks goes to \textbf{Thomas Klausner} who helped in writing and reviewing our
man pages. Thanks to \textbf{Petra Zeidler} for administering the GSoC program
for \textit{The NetBSD Foundation}.
\begin{thebibliography}{20}
\bibitem{1}
\emph{NetBSD manual page for apropos(1)}
\newline http://netbsd.gw.com/cgi-bin/man-cgi?apropos++NetBSD-5.1

\bibitem{2}
\emph{NetBSD manual page for makewhatis(1)}
\newline http://netbsd.gw.com/cgi-bin/man-cgi?makewhatis++NetBSD-5.1

\bibitem{3}
Brin, S.; Page L.
\emph{The Anatomy of a Large-Scale Hypertextual Web Search Engine}
Computer Networks and ISDN Systems
30:107-117, 1998
\bibitem{4}
Manning; Raghwan; Schutze
\emph{Introduction to information retrieval},
3-9,
2008
\bibitem{5}
Porter, M. F.
\emph{An algorithm for suffix stripping},
Program,
14(3): 130-137,
1980

\bibitem{6}
\emph{mdocml online manual page for
libmandoc}
\newline http://mdocml.bsd.lv/mandoc.3.html

\bibitem{7}
\emph{Sqlite home page}
\newline http://sqlite.org

\bibitem{8}
\emph{Sqlite FTS3 and FTS4 Extensions}
\newline  http://sqlite.org/fts3.html

\bibitem{9}
\emph{Online manual page for makemandb(1)}
\newline http://netbsd-soc.sourceforge.net/projects/apropos\_replacement/makemandb.html1

\bibitem{10}
\emph{Online manual page for apropos-utils(3)}
\newline http://netbsd-soc.sourceforge.net/projects/apropos\_replacement/apropos-utils.html3

\bibitem{101}
\emph{man-db, the on-line manual database}
\newline http://man-db.nongnu.org/

\bibitem{12}
\emph{Online manual page for mandocdb(1)}
\newline http://mdocml.bsd.lv/mandocdb.8.html

\bibitem{13}
\emph{NetBSD online manual page for btree(3)}
\newline http://netbsd.gw.com/cgi-bin/man-cgi?btree+3+NetBSD-5.1

\bibitem{14}
Fuhr, Norbert
\emph{Probabilistic models in information
retrieval}
The Computer Journal
1992
\bibitem{15}
Zaragoza H.; Craswell N.; Taylor M.;
Saria S.; Robertson S.
\emph{Microsoft Cambridge at TREC–13: Web
and HARD tracks}
In proceedings of TREC-2004

\bibitem{16}
\emph{Online manual page for man(7)}
\newline http://mdocml.bsd.lv/man.7.html

\bibitem{17}
\emph{Google Summer of Code home page}
\newline http://code.google.com/soc/

\bibitem{18}
\emph {The mdocml project home page}
\newline http://mdocml.bsd.lv
\end{thebibliography}
\end{document}
